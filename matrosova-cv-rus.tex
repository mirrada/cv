\include{header}

% Language-specific personal data
\name{Мария \vspace{0.4ex} \\}{Матросова (Дуженко)\\}
\title{}
\address{Россия, Москва}


\begin{document}
\maketitle
Выпускница ВМК МГУ с опытом разработки, преподавания и чутьём на ошибки. Ищу работу программистом C++, Java или тестировщиком (работа преподавателем у меня уже есть).

\section{Навыки}

\subsection{Языки программирования}

\cvitem{основные}{
	\textbf{C++}: с 2011, разработка вычислительных алгоритмов. \newline
	\textbf{Java}: с 2011 по 2013, разработка текстовых алгоритмов и GUI.
}

\subsection{Библиотеки и технологии}

\cvitem{C++}{
	\textbf{LAPACK}, \textbf{MPICH}, \textbf{OpenMP}, \textbf{STL}: проекты по вычислительным алгоритмам на суперкомпьютере.
}
\cvitem{Java}{
	\textbf{Swing}: разработка простых GUI. \newline
	\textbf{Apache POI}: обработка текстовой информации в документах Word. \newline
	\textbf{JUnit}: интеграционные тесты. 
}

\subsection{Прикладные программы и системы}

\cvitem{основные}{
	\textbf{Windows}, \textbf{Visual Studio}, \textbf{IntelliJ IDEA}, \textbf{SmartGit}, \textbf{SVN}
}
\cvitem{вторичные}{
	\textbf{Unix}, \textbf{\LaTeX}
}


\section{Опыт работы}

\subsection{Основной}

\cvitem{с~2014\cvmonth{09}}{
	\textbf{ФИВТ МФТИ (ГУ)}, \emph{преподаватель}. \newline 
	Проведение семинаров по курсу <<Алгоритмы и структуры данных>> у студентов второго года обучения. Включает в себя изучение широкого класса алгоритмов, структур данных, языка C++.
}

\cvitem{с~2011\cvmonth{09} по~2013\cvmonth{12}}{
	\textbf{ЗАО Эвентос}, \emph{младший разработчик, разработчик}. \newline
	Создание и проектирование системы для синтеза и анализа русских словоформ. Прочие мелкие проекты, так или иначе связанные с анализом текстовой информации. Программирование на языке Java. 
}

\subsection{Вторичный}	

\cvitem{с~2015\cvmonth{04} по~2015\cvmonth{06}}{
	\emph{Редактор}. \newline
	{Вычитка первого тома учебного пособия для студентов <<Математические основы физики>> авторства В.А.Абрамовского, Г.И.Архипова, О.Н.Найды}.  
}

\cvitem{2015\cvmonth{03}}{
	\textbf{Кафедра Математической кибернетики факультета ВМК МГУ}. \newline 
	Подбор задач заочных туров для \myhttplink[Универсиады «Ломоносов»
	по прикладной математике и информатике]{msu.edolymp.ru/?login_required=1}.
}

\cvitem{с~2015\cvmonth{02} по~2015\cvmonth{05}}{
	\textbf{Кафедра Математической кибернетики факультета ВМК МГУ}, \emph{преподаватель}. \newline 
	Прохождение педагогической практики в качестве аспиранта: проведение семинаров по курсу дискретной математики у студентов первого года обучения.
}

\cvitem{с~2013\cvmonth{09} по~2013\cvmonth{12}}{
	\textbf{Кафедра информатики СУНЦ МГУ}, \newline
	\emph{преподаватель-стажёр}. \newline 
	Проведение семинаров по информатике для учащихся 10-11 классов с углублённым изучением биологии и химии.
}

\cvitem{с~2011 по~2013}{
	\textbf{Летние математические школы (<<Kostroma Open>>, <<Умный лагерь>>)}, \newline 
	\emph{помощник преподавателя, вожатый}. \newline 
	Подбор и принятие задач по олимпиадной математике у школьников.	
}

\cvitem{2010\cvmonth{08}}{
	\textbf{Летняя Компьютерная Школа}, \newline 
	\emph{преподаватель, вожатый}. \newline 
	Теоретические лекции и практикум для группы D.	
}


\section{Образование}

\cvitem{с~2009 по~2014}{
	\textbf{Московский Государственный Университет} \newline 
	Факультет Вычислительной Математики и Кибернетики, \newline
	\emph{студент, специалист.} \newline
	Присвоена квалификация <<Математик, системный программист>>. \newline
	Поступила без экзаменов благодаря диплому призёра на \myhttplink[XXI Всероссийской Олимпиаде по Информатике]{neerc.ifmo.ru/school/archive/2008-2009/ru-olymp-roi-2009-standings.html}.
}

\cvitem{с~2013 по~2014}{
	\textbf{Московский Государственный Университет} \newline 
	Факультет Педагогического образования, \emph{студент}. \newline
	Присвоена дополнительная квалификация <<Преподаватель>>.
}

\cvitem{с~2006 по~2008}{
	\textbf{Летняя Компьютерная Школа}, \newline
	\emph{ученица групп С, B, А', А+}. \newline
	Изучение широкого класса алгоритмов и структур данных.
}

\section{Владение языками}

\cvitem{Английский}{Чтение и письмо на произвольные темы. Диалог на технические темы.}
\cvitem{Русский}{Носитель. Грамотное письмо.}



%\renewcommand{\bibliographyitemlabel}{[\arabic{enumiv}]}
%\renewcommand{\refname}{Публикации}
%\bibliographystyle{plain}
%\bibliography{publications}


\end{document}
